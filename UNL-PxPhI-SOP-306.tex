\documentclass[12pt]{unlsilabsop}
\title{Dry air supply}
\date{August 4, 2015}
\author{Frank Meier Aeschbacher}
\approved{Frank Meier Aeschbacher}
\sopid{306}
\sopversion{v0}
\sopabstract{This document describes the regular maintenance for the dry air supply.}
\begin{document}

\maketitle

%------------------------------------------------------------------
\section{Scope}
This document describes the maintenance of the dry air supply. It affects all equipment using dry air (storage cabinet, test stand).

%------------------------------------------------------------------
\section{Purpose}
The cleanroom is a crucial part of the equipment to manufacture modules. It provides a controlled environment for safe handling of modules during manufacturing and storage.

%------------------------------------------------------------------
%\section{Definitions}

%------------------------------------------------------------------
\section{Responsibilities}
Every person handling or testing modules is responsible to maintain the dry air supply. While a minimum schedule for maintenance activities is outlined in this document, any person is allowed to trigger out-of-cycle maintenance if needed to maintain working conditions.

%------------------------------------------------------------------
\section{Equipment}

\begin{itemize}
    \item Titus dryer, model TMDM 1100A
    \item Pressure reducing valve, set to 60\,psi
    \item Data logger, Measurement Computing USB-502-LCD, set to store one data point per minute
\end{itemize}

% Consider adding images of key equipment if it helps

%------------------------------------------------------------------
\section{Procedure}

% Mention to record certain observations if this is needed
\subsection{Weekly maintenance}
\begin{enumerate}
    \item Check setting of pressure reducing walve. Needs to be set to 60\,psi.
    \item Check airflow. Two options:
    \begin{enumerate}
        \item Inside clearoom near the dry storage cabinat, you should hear the permanent air flow
        \item When operating the cold box, the dry air flow get activated. The flow meters will indicate a flow
    \end{enumerate}
    \item Check the data from the data logger.
    \begin{enumerate}
        \item Remove the logger from the dry air cabinet and read it out on a computer.
        \item The humidity reading should be below 5\%, except for the moments when the logger is outside (usually when it needs to be removed to read out). Upload the data to the Elog.
    \end{enumerate}
\end{enumerate}

%------------------------------------------------------------------
\section{Documentation}
The following information needs to be recorded in the report for the UNL logbook:
\begin{itemize}
    \item Date, time (start--end) and operator name.
    \item Any special observations, e.g.~any deviation from expectation.
    \item Add the datafile from the data logger.
    \item Any action taken.
\end{itemize}

\end{document}

