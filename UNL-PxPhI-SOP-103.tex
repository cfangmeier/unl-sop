\documentclass[12pt]{unlsilabsop}
\title{Module assembly: Gluing of HDI to BBM}
\date{September 10, 2014}
\author{Frank Meier Aeschbacher}
\approved{Frank Meier Aeschbacher}
\sopid{103}
\sopversion{v0}
\sopabstract{Describes the procedures to glue HDI on bare modules using the robotic gantry. The procedure takes place in the clean room.}
\begin{document}

\maketitle

%------------------------------------------------------------------
\section{Scope}
This is a regular part in the manufacturing process of pixel modules at UNL.

%------------------------------------------------------------------
\section{Purpose}
Modules require an additional

%------------------------------------------------------------------
\section{Definitions}

%------------------------------------------------------------------
\section{Responsibilities}

%------------------------------------------------------------------
\section{Equipment}

\begin{itemize}
    \item Gantry
    \item Tools placed on rack: grabber and picker tool
    \item Chucks equipped carrier chucks for BBM, HDI, glue reservoir, stamp tools, and weight tools
    \item Araldite
    \item Post-It notes, used as surface to mix Araldite
    \item Spatula
    \item PVC squeegee, disposable
    \item Cleaning tools, consisting of brush and 2-Propanole; to clean glue from stamps and reservoir
    \item Felt tip marker, to write batch number on chuck
    \item Vacuum tool to release BBM from GelPak (if BBM were delivered in GelPak)
\end{itemize}

TODO: add image of setup. Make note that arrangement may vary as long as it matches the setup in the software

%------------------------------------------------------------------
\section{Procedure}

\begin{enumerate}
    \item Handle bare modules only with proper protection: ESD wristband, gloves, face mask.
    \item Perform a readiness check of the gantry:
    \begin{enumerate}
	\item Test presence of vaccum on the gauge, record the pressure.
	\item Start LabView control software. Record version. TODO: note further checks
    \end{enumerate}
    \item Pick parts from storage and identify them. Check that the parts satisfy the quality criteria by retrieving their information in the database.
    \item Assign a new batch number (\texttt{Nxxx}) and assign a unique number (\texttt{Nxxxyy}) according to the rule described in \texttt{SOP~000}.
    \item Write the batch number to the BBM chuck on an edge for further identification.
    \item Place the BBM on the chuck.
    \item Place the HDI on the chuck.
    \item Place at minimum the required number of stamp tools and weight tools on the respective chucks.
    \item Adjust configuration in software to reflect positions in use.
    \item Run pattern recognition step. Check if locations found are sound.
    \item Prepare Araldite:
    \begin{enumerate}
	\item Record batch number of Araldite.
	\item Place a hazelnut-sized amount of Araldite (syringe provides both components at once) on Post-It note.
	\item Mix glue for one minute using the spatula.
	\item Place a portion of the glue on all the required positions of the glue reservoir. Evenly distribute and smoothen the surface using the disposable PVC squeegee.
	\item Coarsely clean spatula, dispose off Post-It note and squeegee.
    \end{enumerate}
    \item Run all program steps in sequence (fiducial finding, apply glue, pick and place HDI, put weights). Supervise progress and stop if needed, especially when modules are at risk.
    \item At the end of the full cycle, remove chuck with stamp tools and clean them thorougly using water and 2-propanole at sink outside the cleanroom. Let them dry and bring back to gantry.
    \item Document all actions even if the curing time may not be over.
    \item Finished modules shall not be handled for at least 2~hours.
    \item Weight tools may be removed using the gantry after 2~hours of curing time but modules need to remain protected from any mechanical stress for a total curing time of 8~hours.
    \item At the discretion of the operator, the remaining curing time can take place off the gantry provided the modules are still placed on the original chuck, on a level surface inside the cleanroom. The storage cabinets are not suitable for this (outgasing of glue). Such removal from the gantry needs to be documented.
\end{enumerate}

%------------------------------------------------------------------
\section{Documentation}
The following information needs to be recorded in the report for the UNL logbook:
\begin{itemize}
    \item Date, time (start--end) and operator name
    \item LabView software: version
    \item Id of parts used:
	\begin{itemize}
	    \item HDI: S/N
	    \item BBM: identification on box plus identification according to naming convention
	    \item UNL batch number
	\end{itemize}
    \item Any special observations, e.g.~damage to parts not already recorded during visual inspection, deviations from normal procedures
\end{itemize}

Purdue database: Status of BBM and HDI need to be updated.

\end{document}

