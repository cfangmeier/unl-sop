\documentclass[12pt]{unlsilabsop}
\title{Storage cabinets}
\date{August 4, 2015}
\author{Frank Meier Aeschbacher}
\approved{Frank Meier Aeschbacher}
\sopid{309}
\sopversion{v1}
\sopabstract{This document describes the regular maintenance for the storage cabinets.}
\begin{document}

\maketitle

%------------------------------------------------------------------
\section{Scope}
This document describes the maintenance of the storage cabinets.

%------------------------------------------------------------------
\section{Purpose}
The storage cabinets are crucial equipment for the production operations, especially the dry air storage cabinet. This SOP describes the maintenance steps for keeping it in working condition.

%------------------------------------------------------------------
%\section{Definitions}

%------------------------------------------------------------------
\section{Responsibilities}
Every person working in the cleanroom is responsible to maintain the storage cabinets in good condition. While a minimum schedule for maintenance activities is outlined in this document, any person is allowed to trigger out-of-cycle maintenance if needed to maintain working conditions.

%------------------------------------------------------------------
\section{Equipment}

\begin{itemize}
    \item Wipe pads, non-dusty tissues
    \item 2-Propanol
\end{itemize}

% Consider adding images of key equipment if it helps

%------------------------------------------------------------------
\section{Procedure}

% Mention to record certain observations if this is needed
\subsection{Weekly maintenance}
\begin{enumerate}
    \item Inspect all the cabinets inside the cleanroom. Remove obvious dirt and clean the surfaces with 2-Propanol on wipes.
    \item Remove unused parts and store it in another place.
\end{enumerate}
Note: The humidity of the air in the dry air storage is checked through the maintenance of the dry air supply (SOP~306).

%------------------------------------------------------------------
\section{Documentation}
The following information needs to be recorded in the report for the UNL logbook:
\begin{itemize}
    \item Date, time (start--end) and operator name.
    \item Any special observations, e.g.~any deviation from expectation.
    \item Any action taken.
\end{itemize}

\end{document}

