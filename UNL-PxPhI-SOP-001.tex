\documentclass[12pt]{unlsilabsop}
\title{Access control}
\date{September 14, 2014}
\author{Frank Meier Aeschbacher}
\approved{Frank Meier Aeschbacher}
\sopid{001}
\sopversion{v1}
\sopabstract{Describes access control to the facilities used to manufacture modules.}
\begin{document}

\maketitle

%------------------------------------------------------------------
\section{Scope}
This covers access to the facilities used at UNL to manufacture modules.

%------------------------------------------------------------------
\section{Purpose}
Raw parts and finished modules need special protection. Access control helps to ensure protection of these parts.

%------------------------------------------------------------------
%>\section{Definitions}

%------------------------------------------------------------------
\section{Responsibilities}
All SiLab team members are required to obey these rules and make sure to not grant access to unauthorized people.

%------------------------------------------------------------------
\section{Procedure}

\subsection{Controlled areas}
The are cosists of
\begin{enumerate}
    \item Lab: Room no. JH174

    Door is equiped with a lock. Keys are restricted to UNLHEP members. The facility is shared with other projects (currently CROP project). The door is allowed to remain unlocked as long as any UNLHEP personnel is present.
    \item Cleanroom: Rom no. JH174C

    This consists of a gowning room and the cleanroom itself. The entrance door is equiped with a card reader for use with the N-card. The electronic lock releases the door to persons with proper access settings stored in the N-card access system.
\end{enumerate}

\subsection{Granting access}
Issuance of keys and granting access via N-card is handled by the UNLHEP business office. Any change to the access list requires permission of a HEP faculty member.


\subsection{Visitors}
Visitors may get access to the cleanroom for the following reasons:
\begin{itemize}
    \item Maintenance personnel (UNL or external). Temporary access is given by accompanying those persons. Advance notice is needed to make proper arrangements.
    \item Guests provided with a tour to our facilities. This needs to be reduced to a minimum and requires permission of a faculty or post-doc. People need to be accompanied permanently by at least one team member.
\end{itemize}
The team member accompanying any visitor is responsible to keep the cleanroom in good order and to protect any parts from damage. Cleanroom rules are to be obeyed by any person entering the room.


%------------------------------------------------------------------
\section{Documentation}
Regular team members do not need to report accessing the facilities. For guests, the following needs to be recorded in the UNL logbook:
\begin{itemize}
    \item Upon receiving a visitor request: date, time (start--end) and names of visitor(s) and their affiliation.
    \item Upon actual visit: ammend the report by a confirmation when the visit has terminated.
\end{itemize}

\end{document}

