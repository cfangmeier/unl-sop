\documentclass[12pt]{unlsilabsop}
\title{Cleanroom maintenance}
\date{September 14, 2014}
\author{Frank Meier Aeschbacher}
\approved{Frank Meier Aeschbacher}
\sopid{301}
\sopversion{v0}
\sopabstract{This document describes the cleanroom and its regular maintenance.}
\begin{document}

\maketitle

%------------------------------------------------------------------
\section{Scope}
This document describes the maintenance of the cleanroom. It affects all persons using this room for any work in the scope of the project.

%------------------------------------------------------------------
\section{Purpose}
The cleanroom is a crucial part of the equipment to manufacture modules. It provides a controlled environment for safe handling of modules during manufacturing and storage.

%------------------------------------------------------------------
\section{Definitions}

%------------------------------------------------------------------
\section{Responsibilities}
Every person is responsible to maintain a clean environment. While a minimum schedule for maintenance activities is outlined in this document, any person is allowed to trigger out-of-cycle maintenance if needed to maintain a clean environment.

%------------------------------------------------------------------
\section{Equipment}

\begin{itemize}
    \item Clean room
    \item Swipe pads
    \item 2-Propanol
    \item Swiffer ``Sweeper wet mop refills with Gain''
    \item Vacuum cleaner
    \item Sticky mats (TODO: mention source)
\end{itemize}

% Consider adding images of key equipment if it helps

%------------------------------------------------------------------
\section{Procedure}

% Mention to record certain observations if this is needed
\subsection{Daily maintenance}
\begin{enumerate}
    \item Needs to be done only if room in use. First person entering room is responsible and puts this on records.
    \item Inspect the cleanroom for any unusual dirt. If spotted, report and trigger a weekly maintenance.
    \item Check sticky mats. If excessive dirt is observed, change mats. Record this action.
\end{enumerate}

\subsection{Weekly maintenance}
\begin{enumerate}
    \item Inspect all tables and equipment surfaces (gantry, wirebonder, probe station etc.). If dirt is visible, clean it using swipe pads and 2-propanole
    \item Clean floor using vacuum cleaner, followed by wet cleaning using Swiffer
    \item Change sticky mats (inside and outside)
    \item Check gowning room. Remove unused items. Using Swiffer to clean floor (may reuse the tissue used inside cleanroom)
\end{enumerate}

%------------------------------------------------------------------
\section{Documentation}
The following information needs to be recorded in the report for the UNL logbook:
\begin{itemize}
    \item Date, time (start--end) and operator name
    \item Any special observations, e.g.~damage to parts not already recorded during visual inspection, deviations from normal procedures
    \item Any action taken.
\end{itemize}

\end{document}

