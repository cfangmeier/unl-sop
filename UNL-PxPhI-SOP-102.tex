\documentclass[12pt]{unlsilabsop}
\title{Module assembly: Deliveries of BBM}
\date{September 14, 2014}
\author{Frank Meier Aeschbacher}
\approved{Frank Meier Aeschbacher}
\sopid{102}
\sopversion{v0}
\sopabstract{Describes the procedures upon receiving of BBM. This includes tests performed, documentation and storage.}
\begin{document}

\maketitle

%------------------------------------------------------------------
\section{Scope}
This is a regular part in the manufacturing process of pixel modules at UNL. This process gets triggered upon delivery of a shipment with BBM.

%------------------------------------------------------------------
\section{Purpose}
BBM are a critical part for manufacturing modules. This procedures ensures the bookkeping of new deliveries and triggers required testing procedures. In case of issues this procedure triggers communication with suppliers.

%------------------------------------------------------------------
%>\section{Definitions}

%------------------------------------------------------------------
\section{Responsibilities}

\begin{itemize}
    \item Shipper: Informs us of any shipment underway (usually Fermilab or RTI is the shipper).
    \item Business office: Informs us of received shipments
    \item SiLab Team member: Picks up shipment and follows instructions. Keeps new deliveries separate from already handled deliveries until this procedure has been finished.
\end{itemize}

%------------------------------------------------------------------
\section{Equipment}

\begin{itemize}
    \item Probe station
    \item Microscope with USB camera attached and illumination
\end{itemize}

%------------------------------------------------------------------
\section{Procedure}

\begin{enumerate}
    \item Unpack delivery in receiving room but keep GelPak boxes closed until transferred into cleanroom. Do not bring cardboard boxes into cleanroom.
    \item Handle BBM only with proper protection: ESD wristband, gloves, face mask.
    \item Open GelPak boxes and get a first impression: Shipment should be in order and no immediate damage is visible.
    \item Take note of identifying information: wafer number, position inside wafer. Record.
    \item Compare list with delivery statement. Report any inconsistency in documentation and inform shipper.
    \item Perform the visual inspection of BBM according to SOP~202.
    \item Perform the IV test of BBM according to SOP~203.
    \item If requested to do so (optional): perform the electrical acceptance test of HDI according to SOP~208.
    \item Failed BBM should be clearly marked on the outside of the box. Shipper needs to be informed and action taken.
    \item Document any findings.
\end{enumerate}
Note: In certain cases delivery comes on dicing tape instead of GelPak boxes. If so, the procedure needs to be changed accordingly. Removal of BBM from dicing tape is a delicate procedure and should only be performed if trained properly.

%------------------------------------------------------------------
\section{Documentation}
The following information needs to be recorded in the report for the UNL logbook:
\begin{itemize}
    \item Date, time (start--end) and operator name
    \item Id of delivery: date received, shipment/tracking number
    \item List of parts received
    \item Any special observations, e.g.~damage to parts, deviations from normal procedures
\end{itemize}
A template document to fill in is provided on the TWiki page and may be used as guidance.

Purdue database: Status of BBM needs to be updated.

\end{document}

