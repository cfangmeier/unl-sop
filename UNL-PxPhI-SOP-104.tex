\documentclass[12pt]{unlsilabsop}
\title{Module assembly: Wirebonding of modules}
\date{September 14, 2014}
\author{Frank Meier Aeschbacher}
\approved{Frank Meier Aeschbacher}
\sopid{104}
\sopversion{v0}
\sopabstract{Describes the procedures to wirebond modules using the Delvotec 56XX wirebonder. The procedure takes place in the clean room.}
\begin{document}

\maketitle

%------------------------------------------------------------------
\section{Scope}
This is a regular part in the manufacturing process of pixel modules at UNL.

%------------------------------------------------------------------
\section{Purpose}
The wirebonds provide the electrical connections between the ROC and the HDI. Additional wirebonds are required to provide the HV bias potential to the sensor.

%------------------------------------------------------------------
%>\section{Definitions}

%------------------------------------------------------------------
\section{Responsibilities}

%------------------------------------------------------------------
\section{Equipment}

\begin{itemize}
    \item Delvotec 56XX wirebonder
    \item Delvotec 5630 bond head
    \item Chuck
    \item Bond wire, aluminium 25\,$\mu$m diameter, Heraeus ALW-29S on 2" spool
    \item Tweezers for wire handling
    \item Test coupon
    \item Optional, in desired amounts: (in case immediate testing is required)
    \begin{itemize}
	\item Module carrier
	\item Flex cable
    \end{itemize}
\end{itemize}

%------------------------------------------------------------------
\section{Procedure}

The procedure below describes the steps for bonding one module. The chuck provides space for up to 4 modules, which may be processed in one batch at the discretion of the operator.
\begin{enumerate}
    \item Handle bare modules only with proper protection: ESD wristband, gloves, face mask.
    \item If not already mounted: Mount the bond head following the manual
    \item Perform a readiness check of the wirebonder:
    \begin{enumerate}
	\item Check if bond wire feeds. If in doubt, re-feed wire according to manual.
	\item If bonding for the first time this day: Make a manual test bond on a coupon. Add this to your report.
	\item Check if vacuum is present by reading the gauge. If not, perform the following checks: vacuum pump is on (outside cleanroom), valve is open.
    \end{enumerate}
    \item Depending on how modules were received, proceed either way:
    \begin{enumerate}
	\item Module handed over on chuck from glueing: Take note of placement of modules for verification.
	\item Bonding of individual modules: Pick modules from storage and identify them (HDI S/N and UNL batch no.). Place them on a chuck, take note which module is placed in with location.
    \end{enumerate}
    \item Check that the parts satisfy the quality criteria by retrieving their information in the database. 
    \item Visually inspect all modules for any possible issues:
    \begin{itemize}
	\item bond pads on HDI and ROC are clean
	\item HV bond pad on sensor is clean and accessible for the bond head
    \end{itemize}
    \item Record the HDI S/N and UNL batch id (\texttt{Nxxxyy}) for each module.
    \item Load the bond program named TODO: insert correct program name
    \item Run alignment step.
    \item Run program. Supervise progress and stop in need, especially when modules are at risk or wirebonds are continuously missing. No stop needed for an occasional miss.
    \item At the end of the full cycle, check visually if all wirebonds are present. Rebond any missing bonds.
    \item Depending on how the production was planned, proceed either way:
    \begin{itemize}
	\item Encapsulation imminent or foreseen in the next days: Keep modules on chuck and put to intermediate storage
	\item Functional test required immediately, encapsulation postponed, or not required: Remove modules from chuck and place on module carriers. If testing is the next step, attach a flex cable.
    \end{itemize}
\end{enumerate}

%------------------------------------------------------------------
\section{Documentation}
The following information needs to be recorded in the report for the UNL logbook:
\begin{itemize}
    \item Date, time (start--end) and operator name
    \item Wirebonder software: version
    \item Id of parts used
    \item Batch number and expiration date of wire
    \item Any special observations, e.g.~damage to parts not already recorded during visual inspection, deviations from normal procedures
\end{itemize}
A template document to fill in is provided on the TWiki page and may be used as guidance.

Purdue database: Status of BBM and HDI need to be updated.

%------------------------------------------------------------------
\section{References}

The manual to the wirebonder is available in the cleanroom as a printed copy in a folder and electronically as PDF on the cleanroom computer.

\end{document}

