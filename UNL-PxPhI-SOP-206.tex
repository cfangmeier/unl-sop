\documentclass[12pt]{unlsilabsop}
\title{Electrical acceptance test of assembled modules}
\date{August 3, 2015}
\author{Frank Meier Aeschbacher}
\approved{Frank Meier Aeschbacher}
\sopid{206}
\sopversion{v0}
\sopabstract{Describes the procedures for the electrical test of final modules. This consists of taking the IV curve and running a so-called \emph{full test} with the standardized test software. Any deviations from normal appearance will be documented.}
\begin{document}

\maketitle

%------------------------------------------------------------------
\section{Scope}
This is a regular test in the manufacturing process of pixel modules at UNL.

%------------------------------------------------------------------
\section{Purpose}
The electrical acceptance test of final modules is a critical step in manufacturing modules. This test is a critical control point to make sure the material we send out meets expectations and to ensure the quality.

%------------------------------------------------------------------
%>\section{Definitions}

%------------------------------------------------------------------
%\section{Responsibilities}

%------------------------------------------------------------------
\section{Equipment}

\begin{itemize}
\item \textbf{Cold box} 
\item \textbf{DTB with adapter cards} 4 each, adapter cards connected via SCSI ribbon cable. Power brick connected. All 4 connected to the computer via separate USB cables. Do not use a USB hub unless you are sure it works.
\item \textbf{Computer} A Linux or Mac~OS~X based computer with pXar, ElCommandante and MoReWeb installed.
\item \textbf{Chiller} TODO: Add model
\item \textbf{Dry air supply} TODO: Add model
\item \textbf{HV power supply} Keithly model 2410, connected to the same computer through a USB to RS232 adapter
\end{itemize}

%------------------------------------------------------------------
\section{Procedure}

\begin{enumerate}
    \item Handle modules only with proper protection: ESD wristband, gloves, face mask.
    \item Make sure the supplies are running:
    \begin{itemize}
        \item Chilled water: The chiller needs to be turned on and set to a temperature of XXX$^\circ$C. Water flow set to XXX\,gal/min.
        \item Dry air: The main valve needs to be open and set to XX\,psi.
        \item All DTB connected to power and LED indicate they are powered.
    \end{itemize}
    \item Turn the cold box on and set it to a temperature of 17$^\circ$C. The temperature need to be stable before running module tests.
    \item Place the modules on the cold box. For this, do:
    \begin{itemize}
        \item Open the lid. The sample area needs to be clean, no dust or dirt visible on cold plate.
        \item Place up to four modules on theit carriers using the positioner pins.
        \item Connect the cable to the adapter cards on the cold box.
        \item Close the lid. Make sure the cables lie flat and the lid makes a good seal.
    \end{itemize}
    \item Run the IV test.

        TODO: explain steps
    \item Run the \emph{full test} using ElCommandante
    \item Document any findings. Upload the following results to the Purdue database:
\end{enumerate}

%------------------------------------------------------------------
\section{Documentation}
If a module passes the tests, a status update in the database needed. Go to ``MAIN MENU''$\rightarrow$``Part List'' and select your module. Upload any pictures as needed and add comments as needed.

No need to make an entry in the Elog if nothing special happened.

\end{document}

