\documentclass[12pt]{unlsilabsop}
\title{Protective measures}
\date{August 6, 2015}
\author{Frank Meier Aeschbacher}
\approved{Frank Meier Aeschbacher}
\sopid{003}
\sopversion{v1}
\sopabstract{Describes any measures to protect the cleanroom and modules from contamination.}
\begin{document}

\maketitle

%------------------------------------------------------------------
\section{Scope}
This covers access to the facilities used at UNL to manufacture modules.

%------------------------------------------------------------------
\section{Purpose}
Raw parts and finished modules need special protection. In order to do this, special protective measures are in place.

%------------------------------------------------------------------
%>\section{Definitions}

%------------------------------------------------------------------
\section{Responsibilities}
All SiLab team members are required to obey these rules.

%------------------------------------------------------------------
\section{Materials}
\begin{itemize}
    \item Cleanroom coat
    \item Shoe cover (``booties'')
    \item Hairnet
    \item Facemask
    \item Gloves
    \item ESD strap
\end{itemize}

%------------------------------------------------------------------
\section{Levels of protection}

\begin{itemize}
    \item \textbf{Basic level:} This is the minimal protection for entering the cleanroom. For this level, people entering the cleanroom are required to wear
    \begin{itemize}
	\item Cleanroom coat
	\item Shoe cover
	\item Hairnet
    \end{itemize}
    Persons entering the cleanroom are required to tap each foot on the sticky mats inside and outside the room for at least two times, adding up to four taps per foot.
    \item \textbf{Module handling:} To work with modules, the following additional protective measures need to be used:
    \begin{itemize}
	\item Facemask
	\item Gloves
	\item ESD strap
    \end{itemize}
    Module handling includes (but is not limited to) performing any manufacturing or test step on BBM, HDI or modules not mounted on covered module carriers. People present in the cleanroom not wearing these protective measures are required to keep a safe distance to the module handling area of at least about 1\,m (approx. 3 feet).
\end{itemize}

\end{document}

