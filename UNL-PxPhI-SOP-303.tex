\documentclass[12pt]{unlsilabsop}
\title{Wire bonder and pull tester}
\date{August 6, 2015}
\author{Frank Meier Aeschbacher}
\approved{Frank Meier Aeschbacher}
\sopid{303}
\sopversion{v1}
\sopabstract{This document describes the regular maintenance for the wire bonder and pull tester.}
\begin{document}

\maketitle

%------------------------------------------------------------------
\section{Scope}
This document describes the maintenance of the wire bonder and pull tester.

%------------------------------------------------------------------
\section{Purpose}
The wire bonder and pull tester is a crucial equipment for fabricating and testing modules. This SOP describes the maintenance steps for keeping it in working condition.

%------------------------------------------------------------------
%\section{Definitions}

%------------------------------------------------------------------
\section{Responsibilities}
Every person using the wire bonder and/or pull tester is responsible to maintain it. While a minimum schedule for maintenance activities is outlined in this document, any person is allowed to trigger out-of-cycle maintenance if needed to maintain working conditions.

%------------------------------------------------------------------
\section{Equipment}

\begin{itemize}
    \item Wipe pads, non-dusty tissues
    \item 2-Propanol
    \item Bond wedges
    \item Copper screw
\end{itemize}

% Consider adding images of key equipment if it helps

%------------------------------------------------------------------
\section{Procedure}

% Mention to record certain observations if this is needed
\subsection{Weekly maintenance}
\begin{enumerate}
    \item Clean the surface of the wirebonder using 2-Propanol and wiping pads. Especially check that the surfaces of the chucks are clean and free from any dirt (to produce a good vacuum seal).
    \item Remove any obvious dirt, especially loose pieces of wire.
    \item Inspect the wedge for any obvious damage or dirt. Maybe make test bonds on a coupon. Change in doubt. 
    \item Mount the pull tester and inspect the hook with the microscope for any obvious damage. Replace the hook if needed.
\end{enumerate}

\subsection{Special maintenance}
\begin{enumerate}
    \item Whenever a wedge needs to be replaced, document it. The copper screw should be tightened using a torque wrench, set to 18\,Ncm and use a 0.9\,mm wrench.
    \item The copper screw used to fix the bond wedge is soft. After four mounting-cycles it needs to be replaced. If a wedge gets replaced, check how many times the screw has been used. If it is hitting the lifecycle, replace it and document this.
\end{enumerate}

%------------------------------------------------------------------
\section{Documentation}
The following information needs to be recorded in the report for the UNL logbook:
\begin{itemize}
    \item Date, time (start--end) and operator name.
    \item Any special observations, e.g.~any deviation from expectation.
    \item Any action taken.
\end{itemize}

\end{document}

